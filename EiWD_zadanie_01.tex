\documentclass[11pt]{report}

\usepackage[polish]{babel}
\usepackage{csquotes}
\DeclareQuoteAlias[]{dutch}{polish}
\usepackage{polski}
\usepackage{geometry}
\usepackage{listings}
\usepackage{xurl}
\usepackage{hyperref}
\usepackage{import}
\usepackage{multirow}
\usepackage{titling}
\usepackage{anyfontsize}
\usepackage[htt]{hyphenat}
\usepackage{xcolor}

\newcommand{\prowadzacy}{prof. dr hab. inż. Vasyl Martsenyuk}
\newcommand{\temat}{Ustalenia platformy Jupyter. Użycia biblioteki pandas w celu eksploracji i wizualizacji danych}
\newcommand{\wariant}{1}
\newcommand{\kierunek}{Informatyka}
\newcommand{\stopien}{II}
\newcommand{\tryb}{niestacjonarne}
\newcommand{\semestr}{III}
\newcommand{\grupa}{1}
\newcommand{\nrLab}{1}
\newcommand{\repozytorium}{https://github.com/prybka82-student/eksploracja_i_wizualizacja_danych}

\definecolor{codegreen}{rgb}{0,0.6,0}
\definecolor{codegray}{rgb}{0.5,0.5,0.5}
\definecolor{codepurple}{rgb}{0.58,0,0.82}
\definecolor{backcolour}{rgb}{0.95,0.95,0.92}

\lstdefinestyle{mystyle}{
    backgroundcolor=\color{backcolour},   
    commentstyle=\color{codegreen},
    keywordstyle=\color{magenta},
    numberstyle=\tiny\color{codegray},
    stringstyle=\color{codepurple},
    basicstyle=\ttfamily\footnotesize,
    breakatwhitespace=false,         
    breaklines=true,                 
    captionpos=b,                    
    keepspaces=true,                 
    numbers=left,                    
    numbersep=5pt,                  
    showspaces=false,                
    showstringspaces=false,
    showtabs=false,                  
    tabsize=2
}

\lstset{style=mystyle}

\makeatletter
\renewcommand{\@makechapterhead}[1]{%
\vspace*{0 pt}%
{\setlength{\parindent}{0pt} \raggedright \normalfont
\bfseries\Huge\thechapter.\ #1
\par\nobreak\vspace{40 pt}}}
\makeatother

\newcounter{zadanie}
\newcommand{\zadanie}{
    \refstepcounter{zadanie}
    \filbreak\vspace*{1cm}
    {\noindent\raggedright\Large \textbf{Zadanie~\thezadanie}}
    \vspace{10 pt}\nopagebreak[1]
}

\begin{document}

\title{Eksploracja i wizualizacja danych}
\author{Piotr Rybka}
\date{26.09.2021}

\import{}{EiWD_title_page.tex}

\chapter*{Polecenie}

Zadanie dotyczy pobrania danych z pliku, tworzenia ramki danych, wykonania
poszczególnych zadań na podstawie odpowiedniego zbioru danych:\\

{
    \raggedright\noindent\url{http://ghdx.healthdata.org/record/ihme-data/development-assistance-health-database-1990-2020}
}

\zadanie

Załadować bibliotekę \texttt{pandas}.

\begin{lstlisting}[language=Python]
import pandas as pd
\end{lstlisting}

\zadanie

Utworzyć ramkę danych ze słownika. 

\begin{lstlisting}[language=Python]
dane1 = {
    "liczby": [532, 2532, 523, 2543, 235, 231, 34, 23552, 5324],
    "kategorie": ["a", "b", "c", "d", "e", "f", "g", "h", "i"]
}

df1 = pd.DataFrame(dane1)
df1
\end{lstlisting}

\zadanie 

Zapisać ramkę danych w pliku \texttt{csv}.

\begin{lstlisting}[language=Python]
sciezka = r"C:\Users\piotr\Downloads\dane_ze_slownika.csv"

df1.to_csv(sciezka, encoding="utf-8")
\end{lstlisting}

\zadanie

Utworzyć ramkę danych z listy list.

\begin{lstlisting}[language=Python]
dane2 = [
    [532, 2532, 523, 2543, 235, 231, 34, 23552, 5324],
    ["a", "b", "c", "d", "e", "f", "g", "h", "i"]
]

df2 = pd.DataFrame(dane2)
\end{lstlisting}

\zadanie

Wczytać dane z pliku \texttt{csv}.

\begin{lstlisting}[language=Python]
sciezka = r"C:\Users\piotr\Downloads\IHME_DAH_DATABASE_1990_2020_CSV_1\IHME_DAH_DATABASE_1990_2020_Y2021M09D22.CSV"

df = pd.read_csv(sciezka, low_memory=False, )
\end{lstlisting}

\zadanie

Wyświetlić pierwsze 10 wierszy ramki danych. 

\begin{lstlisting}[language=Python]
df.head(10)
\end{lstlisting}

\zadanie 

Wyświetlić ostatnie 10 wierszy ramki danych.

\begin{lstlisting}[language=Python]
df.tail(10)
\end{lstlisting}

\zadanie

Wyświetlić informacje o ramce danych.

\begin{lstlisting}[language=Python]
df.info()
\end{lstlisting}

\zadanie 

Przetransponować dane w ramce danych. 

\begin{lstlisting}[language=Python]
df_transposed = df.T

df_transposed.head()
\end{lstlisting}

\zadanie 

Wyświetlić liczbę wierszy i kolumn zawartych w ramce danych. 

\begin{lstlisting}[language=Python]
rows, cols = df.shape
print(f"Kolumn: {cols}")
print(f"Wierszy: {rows}")
\end{lstlisting}

\zadanie

Wyodrębnić wiersze i kolumny przy użyciu nazw i indeksów. 

\begin{lstlisting}[language=Python]
df["year"]

df.year

df[["year", "source", "recipient_country"]]

df.loc[:, "year":"channel"] # wszystkie wiersze, kolumny od "year" do "channel" wlacznie

df.iloc[0:10, 2:4] # wiersze od pierwszego do dziesiatego, kolumny od trzeciej do czwartej
\end{lstlisting}

\zadanie 

Wyświetlić podstawowe informacje statystyczne o kolumnach liczbowych (liczba wartości niepowtarzalnych, średnia, odchylenie standardowe, minimum, maksimum, wartości kwartyli).

\begin{lstlisting}[language=Python]
liczbowe = df.select_dtypes(include='number')
liczbowe.describe()
\end{lstlisting}

\zadanie

Wyświetlić podstawowe informacje statystyczne o kolumnach kategoryzowanych (liczba wartości niepowtarzalnych, wartość najczęstsza, liczba wystąpień wartości najczęstszej).

\begin{lstlisting}[language=Python]
kategoryzowane = df.select_dtypes(exclude='number')
kategoryzowane.describe()
\end{lstlisting}

\zadanie

Usunięcie brakujących wartości z ramki danych.

\begin{lstlisting}[language=Python]
df.dropna(inplace=True) # inplace=True - zmodyfikowane zostana oryginalne dane
\end{lstlisting}

\zadanie

Wybrać te wiersze ramki danych, które spełniają podany warunek dotyczący wybranej kolumny.

\begin{lstlisting}[language=Python]
df[df["recipient_country"] == "Poland"]
\end{lstlisting}

\zadanie

Wybrać wiersze ramki danych, które spełniają kilka warunków jednocześnie.

\begin{lstlisting}[language=Python]
df[(df["recipient_isocode"] == "POL") & (df["year"] >= 2005)]
\end{lstlisting}

\zadanie

Wybrać wiersze, które zawierają w kolumnie kategoryzowanej określone słowo.

\begin{lstlisting}[language=Python]
df[df["source"]=="Japan"]
\end{lstlisting}

\zadanie

Wybrać wiersze, które nie zawierają w kolumnie kategoryzowanej podanego wyrazu.

\begin{lstlisting}[language=Python]
selection = df[df["gbd_region"]!="Asia, East"]
\end{lstlisting}

\zadanie

Utworzyć kolumnę na podstawie istniejącej kolumny w ramce danych.

\begin{lstlisting}[language=Python]
recipient_countries = df.recipient_country
\end{lstlisting}

\zadanie

Usunąć kolumnę z ramki danych.

\begin{lstlisting}[language=Python]
df_copy = df
df_copy.drop(["source", "channel"], axis=1, inplace=True) # usuwa kolumny "source" i "channel"
\end{lstlisting}

\zadanie

Zmienić nazwę kolumny w ramce danych.

\begin{lstlisting}[language=Python]
df_copy.rename(columns={"year": "rok", "recipient_isocode": "kod_iso_odbiorcy", "recipient_country": "kraj_odbiorcy"}, inplace=True)
\end{lstlisting}

\zadanie

Zapisać ramkę danych jako plik w formacie \texttt{csv}.

\begin{lstlisting}[language=Python]
sciezka = r"C:\Users\piotr\Downloads\df_copy.csv"
df.to_csv(sciezka, encoding="utf-8")
\end{lstlisting}

\zadanie

Wyświetlić średnią, maksymalną i minimalną wartość danej kolumny.

\begin{lstlisting}[language=Python]
col = df["elim_ch"]
mean = col.mean()
max_ = col.max()
min_ = col.min()

print(f"srednia: {mean}\nmaksimum: {max_}\nminimum: {min_}")
\end{lstlisting}

\zadanie

Podać liczbę wierszy wybranej kolumny.

\begin{lstlisting}[language=Python]
df.rok.count()
\end{lstlisting}

\zadanie

Wyodrębnić wartości unikatowe w kolumnie ramki danych.

\begin{lstlisting}[language=Python]
df_copy["kraj_odbiorcy"].unique()
\end{lstlisting}

\zadanie

Wyświetlić liczbę rekordów spełniajacych podany warunek. 

\begin{lstlisting}[language=Python]
df_copy[df_copy["kraj_odbiorcy"]=="Turkey"].rok.count()
\end{lstlisting}

\zadanie

Posortować wiersze ramki danych według wartości w wybranej kolumnie (malejąco i rosnąco).

\begin{lstlisting}[language=Python]
df_copy.sort_values(['kraj_odbiorcy'], ascending=True).head()

df_copy.sort_values(['kraj_odbiorcy'], ascending=False).head()
\end{lstlisting}

\zadanie

Wyświetlić wiersze zawierające 10 największych i najmniejszych wartości z wybranej kolumny.

\begin{lstlisting}[language=Python]
df.nlargest(10, 'elim_ch')[['rok', 'elim_ch']]

df.nsmallest(10, 'elim_ch')[['rok', 'elim_ch']]
\end{lstlisting}

\zadanie

Wyświetlić wiersze zawierające 10 największych wartości wybranej kolumny pod warunkiem, że wartości w innej kolumnie mają określoną wartość.

\begin{lstlisting}[language=Python]
df[(df['kraj_odbiorcy'].isin(['Poland', 'Germany'])) & (df['rok'] == 1990)].nlargest(10, 'elim_ch')
\end{lstlisting}

\zadanie

Pogrupować wiersze według wartości kolumny kategoryzowanej, a następnie uśrednić wartości wszystkich kolumn liczbowych.

\begin{lstlisting}[language=Python]
df.groupby('kraj_odbiorcy').agg('mean')
\end{lstlisting}

\zadanie

Pogrupować wiersze wg wartości kolumny kategoryzowanej, a następnie uśrednić wartości dla wybranych kolumny, a dla innych obliczyć częstość wystąpień i medianę.

\begin{lstlisting}[language=Python]
kraje = df.groupby('kraj_odbiorcy').agg({'gbd_location_id': ['mean'], 'elim_ch': ['mean'], 'prelim_est': ['mean'], 'rok': ['count', 'median'], 'wb_location_id': ['count', 'median'], 'gbd_superregion_id': ['count', 'median']})
\end{lstlisting}

\zadanie

Pobrać nazwy kolumn indeksu złożonego.

\begin{lstlisting}[language=Python]
kraje.index
\end{lstlisting}

\zadanie

Posortować kolumnę indeksu złożonego.

\begin{lstlisting}[language=Python]
kraje['elim_ch']['mean'].sort_values(ascending=False)
\end{lstlisting}

\zadanie

Stworzyć tabelę przestawną (ang. \textit{pivot table}) na podstawie ramki danych.

\begin{lstlisting}[language=Python]
pivot = df.pivot_table(values='elim_ch', index='kraj_odbiorcy', columns='rok', aggfunc='count', margins=False, dropna=True, fill_value=None)
\end{lstlisting}

\zadanie

Wyświetlić indeksy i kolumny tabeli przestawnej.

\begin{lstlisting}[language=Python]
pivot.index
\end{lstlisting}

\zadanie

Utworzyć indeks złożony tabeli przestawnej i wyświetlić jego zawartość.

\begin{lstlisting}[language=Python]
pivot2 = df.pivot_table(values='elim_ch', index=['gbd_region', 'kraj_odbiorcy'], columns='rok', aggfunc='count', margins=False, dropna=True, fill_value=None)
\end{lstlisting}

\zadanie

Zaimportować moduł \texttt{pyplot} z biblioteki \texttt{matplotlib} oraz wskazać, że wykresy należy rysować bezpośrednio w zeszycie, a nie w osobnej zakładce.

\begin{lstlisting}[language=Python]
import matplotlib.pyplot as plt

%matplotlib inline
\end{lstlisting}

\zadanie

Wyświetlić wykres na podstawie tabeli przestawnej.

\begin{lstlisting}[language=Python]
pivot3 = df.pivot_table(values='elim_ch', index='rok', columns='gbd_superregion', aggfunc='mean', margins=False, dropna=True, fill_value=None)

fig = pivot3.plot(kind='line')
plt.ylabel('mean value of elim_ch')
plt.title('Mean values of elim_ch in superregions per year')
plt.rcParams["figure.figsize"] = (20,10)
\end{lstlisting}

\zadanie

Narysować histogram na podstawie wartości w wybranej kolumnie.

\begin{lstlisting}[language=Python]
df_bar = df.pivot_table(values='elim_ch', index='rok', columns='gbd_superregion', aggfunc='sum', margins=False, fill_value=None, dropna=True)
df_bar.plot(kind='bar')
plt.ylabel('elim_ch')
plt.title('elim_ch by year and superregion')
\end{lstlisting}

\zadanie

Przedstawić sposoby łączenia ramek danych za pomocą metod \texttt{merge} i \texttt{concat}.

\begin{lstlisting}[language=Python]
part1 = df[['rok', 'kraj_odbiorcy', 'elim_ch', 'gbd_region']]
part2 = df[['rok', 'kraj_odbiorcy', 'prelim_est', 'gbd_superregion']]

pd.merge(part1, part2, on = ['rok', 'kraj_odbiorcy'], how='inner').head()

pd.merge(part1, part2, on = [part1.index, part2.index], how='inner').head()

pd.concat([part1, part2], axis=0)

pd.concat([part1, part2], axis=0).shape

pd.concat([part1, part2], axis=1)

pd.concat([part1, part2], axis=1).shape
\end{lstlisting}

\zadanie

Pokazać dodawanie nowych kolumn za pomocą operacji matematycznych.

\begin{lstlisting}[language=Python]
df['sum'] = df['elim_ch'] + df['prelim_est']
\end{lstlisting}

\zadanie

Przedstawić przykładowe dodawanie nowych kolumn przy użyciu wyrażenia lambda.

\begin{lstlisting}[language=Python]
df['years_ago'] = df['rok'].apply(lambda y: 2021 - int(y))
\end{lstlisting}

\zadanie

Ukazać możliwości pracy z dużymi plikami przy użyciu argumentu \texttt{chunksize}.

\begin{lstlisting}[language=Python]
sciezka = r"C:\Users\piotr\Downloads\IHME_DAH_DATABASE_1990_2020_CSV_1\IHME_DAH_DATABASE_1990_2020_Y2021M09D22.CSV"

chunks = pd.read_csv(sciezka, low_memory=False, chunksize=100_000)

for i, chunk in enumerate(chunks):
    print(f"\n\nChunk number {i+1}:")
    print(chunk.iloc[0:3,0:4])
\end{lstlisting}

\chapter*{Wnioski}

\begin{itemize}
    \item Załadowanie biblioteki \texttt{pandas}: \texttt{import pandas as pd}.
    \item Tworzenie ramki danych ze słownika: \texttt{pd.DataFrame({"kol1": [], "kol2": []})}.    
    \item Tworzenie ramki danych z listy list: \texttt{pd.DataFrame([[], [], []])}.
    \item Wczytanie danych do ramki danych z pliku \texttt{csv}: \texttt{pd.read\_csv(sciezka, low\_memory=False)}.
    \item Zapis danych z ramki danych do pliku \texttt{csv}: \texttt{pd.to\_csv(sciezka, encoding="utf-8")}.
    \item Wybranie \textit{n} pierwszych lub ostatnich wierszy ramki danych:
    \begin{itemize}
        \item \texttt{df.head(n)};
        \item \texttt{df.tail(n)};
    \end{itemize}
    \item Transponowanie danych (zamiana miejscami kolumn i wierszy): \texttt{df.T}.
    \item Zestawienie informacji o ramce danych:
    \begin{itemize}
        \item \texttt{df.info()} -- spis kolumn wraz z podaniem typu i liczbą wartości niepustych (\textit{non-null});
        \item \texttt{rows, cols = df.shape} -- liczba, odpowiednio, wierszy i kolumn;
        \item \texttt{df.describe()} -- miary statystyczne opisujące dane;
    \end{itemize}
    \item Sposoby wybrania kolumny danych:
    \begin{itemize}
        \item \texttt{df['nazwa\_kolumny']};
        \item \texttt{df.nazwa\_kolumny};
        \item \texttt{df[["kolumna1", "kolumna2", "kolumna3"]]};
        \item \texttt{df.loc[:, "kolumna1":"kolumnaX"]} -- wszystkie wiersze kolumn od 1. do $x$;
        \item \texttt{df.iloc[:, 2:4]} -- wszystkie wiersze kolumn od 3. do 4.
    \end{itemize}
    \item Odfiltrowanie kolumn o wartościach liczbowych: \texttt{df.select\_dtypes(include='number')} i kategoryzowanych: \texttt{df.select\_dtypes(exclude='number')}.
    \item Odfiltrowanie wierszy spełniających podane warunki: 
    \begin{itemize}
        \item \texttt{df[df["{}nazwa kolumny"] == "wartosc"]};
        \item \texttt{df[(df["{}nazwa\_kolumny"] == "wartosc") \& (df["kolumnaX"] isin(["wart1", "wart2"])) \& (df["{}nazwa\_kolumny"{}] >= 100)]};
    \end{itemize}
    \item Usunięcie brakujących wartości z ramki danych: \texttt{df.dropna(inplace=True)}.
    \item Usunięcie kolumny o wskazanych nazwach: \texttt{df.drop(["kolumna1", "kolumna2"], axis=1, inplace=True)}.
    \item Podstawowe miary opisujące dane: 
    \begin{itemize}
        \item liczba wartości: \texttt{df.kolumna.count()};
        \item średnia arytmetyczna: \texttt{df.kolumna.mean()};
        \item maksimum: \texttt{df.kolumna.max()};
        \item minimum: \texttt{df.kolumna.min()}.
    \end{itemize}
    \item Odfiltrowanie wartości niepowtarzalnych z kolumny: \texttt{df.kolumna.unique()}.
    \item Sortowanie danych wg danych we wskazanej kolumnie: \texttt{df.sort\_values(["kolumna"], ascending=True)}.
    \item Wyświetlenie $n$ najmniejszych i największych wartości we wskazanych kolumnach: 
    \begin{itemize}
        \item \texttt{df.nlargest(n, "kolumna")};
        \item \texttt{df.nsmallest(n, "kolumna")}.
    \end{itemize}
    \item Grupowanie i agregowanie danych: \texttt{df.groupby("kolumna").agg('mean')}.
    \item Indeks pogruowanych wartości: \texttt{df.index}.
    \item Tabela przestawna z indeksem prostym: \texttt{pivot1 = df.pivot\_table(values="wartosci", index=["kol1"], columns="rok", aggfunc="count", margins=False, dropna=True, fill\_value=None)}
    \item Załadowanie modułu \texttt{pyplot}: \texttt{import maptplotlib.pyplot as plt}. Opcja rysowania wykresów bezpośrednio w zeszycie: \texttt{\%matplotlib inline}.
    \item Wykres na podstawie tabeli przestawnej: \texttt{df.pivot\_table(values="kol1", index="kol2", aggfunc="mean", margins=False, dropna=True, fill\_value="{}none")}
    \item Histogram na podstawie tabeli przestawnej: \texttt{df.pivot\_table(values="kol1", index="kol2", columns\_gbd)}.
    \item Łączenie danych: 
    \begin{itemize}
        \item  \texttt{merge}: \texttt{pd.merge(frame1, frame2), on=[part1.index, part2.person], how='inner'}
        \item \texttt{concat}: \texttt{pd.concat(frame1, frame2), axis=0}
    \end{itemize}
    \item Dodawanie kolumn: \texttt{df["{}nowaKolumna"] = df["kol1"] + df["kol2"]}.
    \item Podział ładowanych danych na porcje o wielkości $x$ bajtów: \texttt{pd.read\_csv(sciezka, low\_memory=False, chunksize=100\_000)}). 
\end{itemize}

\end{document}